\documentclass[11pt]{article}

\usepackage[utf8]{inputenc}

\usepackage[a4paper, total={7.5in, 10in}]{geometry}
%\usepackage{fullpage,amsfonts,mathpazo}
\usepackage{amsmath,amsthm,amssymb,amsfonts}
\usepackage{bbm}
\usepackage{mathtools}
\usepackage{easybmat}
\usepackage{bm}
\usepackage{pdfpages}
\usepackage{listings}
\usepackage{amsmath,graphicx,centernot}
\usepackage{longtable}

\newcommand{\N}{\mathbb{N}}
\newcommand{\Z}{\mathbb{Z}}
\newcommand{\Q}{\mathbb{Q}}
\newcommand{\R}{\mathbb{R}}


\newtheorem{thm}{Theorem}[section]
\newtheorem{lem}[thm]{Lemma}

\def\Inf{\operatornamewithlimits{inf\vphantom{p}}}

\newcommand\ind{\protect\mathpalette{\protect\independenT}{\perp}}
\def\independenT#1#2{\mathrel{\rlap{$#1#2$}\mkern2mu{#1#2}}}

\DeclarePairedDelimiter\floor{\lfloor}{\rfloor}
\newcommand{\nind}{\centernot{\ind}}
\newcommand{\specialcell}[2][c]{%
  \begin{tabular}[#1]{@{}l@{}}#2\end{tabular}}


\begin{document}
{\LARGE OptumInsight Data}

\begin{center}
\begin{longtable}{| p{2.5cm} | p{14cm} |} 
    \hline
    Notation & Variable definition \& Characteristics \\  
    \hline
    $i = 1, \ldots, N$ & \specialcell{ Index for the $i$-th patient; \\ $N$ = total number of patients = 14595 } \\
    \hline
    $T_{i0}$ & Index VTE date\\
    \hline
    $(T_{ij}, Y_{ij})$ & \specialcell{$T_{ij}$ = the time of AC prescription; $T_{ij} \geq T_{i0} \forall j$; \\ $Y_{ij}$ = the prescribed anticoagulant at time $T_{ij}$, \\ \ where $j = 1, \ldots, n_i$ with $n_i$ being the number of anticogulant prescriptions after \\ \  index VTE date, and \\ \ $Y_{ij} \in A :=$ \{ DOACS (4 subcategories), LMWH, Warfarin, Other \}} \\
    \hline 
    $(Y_{i1}^*, Y_{i2}^*)$ & \specialcell{ (Index AC, AC at 3 months) \\ \ $Y_{i1}^* \in \{0,1\}^K$, where $K = 7 = |A|$. \\ \ Index AC is defined as the first AC after index VTE date; \\ \ AC at 3 months is defined as the most recent AC prior to index VTE date + \\ \ 90 days, and if a patient stopped on AC in the three months, it is recorded as \\ \ "Not captured" (Warfarin + 60 days, INR + 42 days, \\ \ LMWH/DOACS/Other + 30 days)} \\
    \hline
    $(T_{i1}^*, T_{i2}^*)$ & \specialcell{ (Time of index AC, Time of AC at 3 months) } \\
    \hline
    $\bm V_{ij}$ & A vector of: copay, type of insurance, and \textcolor{blue}{ provider information (TBD)} at $T_{ij}$\\
    \hline 
    $(T_{il}^{(L)}, \bm X_{il}^{(L)})$ & \specialcell{ $T_{il}^{(L)}$ = Time of lab tests \\ \ $\bm X_{il}^{(L)}$ = A vector of: lab test type (hemoglobin, platelets, or GFR), and test \\ \ result at time $T_{il}^{(L)}$, where $l = 1, \ldots, L_i$. \\ \ Note that a majority of patients do not have lab test records.} \\
    \hline
    $D_{ij}$ & Days of supply of the AC at $T_{ij}$; $D_{ij} \in \{1, \ldots, D_{\mathrm{max}} \}$,  where $D_{\mathrm{max}} = 90$. \\
    \hline 
    $S_{ij}$ & Dose of the AC at $T_{ij} = \dfrac{\mathrm{quantity}}{\mathrm{D_{ij}}} \times$ strength; $S_{ij} \in \R_+$ \\
    \hline
    $(T_{ir}^{(I)}, \bm I_{ir})$ & $T_{ir}^{(I)}$ = time of the $r$-th INR test; $I_{ir}$ = INR result at $T_{ir}^{(I)}$, where $r = 1, \ldots, R_i$. \\
    \hline
    \specialcell{ $(T_{ik}^{(a)}, E_{ik},$ \\ \ $\bm C_{ik}, P_{ik})$ } & \specialcell{ $T_{ik}^{(a)}$ = the $k$-th admission date, $T_{ik}^{(a)} \geq T_{i0}$ \\ $E_{ik}$ = length of the $k$-th stay in days, $E_{ik} \geq 1$ \\ $\bm C_{ik}$ = a binary vector indicating the ICD-9 codes associated with the admission; \\ \ $\bm C_{ik} \in \{0,1\}^M$ where $M=$ the number of (tree-structured) ICD-9 codes \\ $k = 1,\ldots, K_i$, where $K_i = $ the number of admissions; \\ \textcolor{blue}{$P_{ik} = $ place of service (POS) (TBD)} } \\
    \hline\hline
    $\bm X_i^{(1)}$ & All time-variant covariates of the above, i.e. $\bm V_{ij}, j=1,\ldots, n_i$; $(T_{il}^{(L)}, \bm X_{il}^{(L)}), l=1,\ldots,L_i$; $D_{ij}$; $S_{ij}$; $(T_{ir}^{(I)}, \bm I_{ir}), r = 1, \ldots, R_i$; $(T_{ik}^{(a)}, E_{ik}, \bm C_{ik}, P_{ik}), k = 1,\ldots, K_i$. \\
    \hline\hline
    $\bm X_i^{(2)}$ & All time-invariant covariates of the $i$-th patient: 
    \begin{enumerate}
        \item index VTE date
        \item index cancer type
        \item index cancer date
        \item gender
        \item SES: education, occupation, division, race, federal poverty level, home ownership, income range, networth range
        \item indicator for having a surgery within 30 days prior to index VTE date
        \item indicator for smoking within 30 days prior to index VTE date
        \item \textcolor{blue}{ place of service associated with index VTE date }
    \end{enumerate} \\
    \hline
\end{longtable}
\end{center}

Notes:
\begin{itemize}
    \item Provider level information colored in \textcolor{blue}{blue} is unidentifiable yet.
    \item Lab tests other than INR tests, i.e. hemoglobin, platelets, and GFR, can be either time-variant or time-invariant. If all records of such lab tests are considered, then they are denoted as $(T_{il}^{(L)}, \bm X_{il}^{(L)})$. If only the most recent lab tests within 30 days prior to index VTE date is considered, then they will go into the time-invariant covariate vector $\bm X_i$.
    \item Since ICD-9 codes are tree-structured, all diagnoses are tree-structured.
\end{itemize}

Scientific question: 
\begin{enumerate}
    \item Predicting the distribution of index AC and AC at 3 months given all covariates: 
    $$ \left[ Y_{i1}^*, Y_{i2}^* \middle| \bm X_i^{(1)}, \bm X_i^{(2)} \right], $$
    where $[A | B]$ denotes the conditional distribution of $A$ given $B$.
    \item Predicting the anticoagulant prescription pattern after index VTE date:
    $$ \left[ Y_{i1},\ldots, Y_{in_i} \middle| \bm X_i^{(1)}, \bm X_i^{(2)} \right]. $$
\end{enumerate}

Features of the data:
\begin{enumerate}
    \item Repeated multivariate outcomes: multiple drugs are prescribed repeatedly
    \item Semi-regular time points: days of supply are commonly 30 days; less common are 15 days and 90 days, etc. Days of supply predict the time of the next prescription reasonably well.
    \item Interrupted time points: information is always lost during hospitalization periods.
\end{enumerate}


\newpage
\textbf{Random thoughts} \\

\begin{itemize}
\item For predicting the anticoagulant pattern with covariates: Titsias, Michalis K., Christopher C. Holmes, and Christopher Yau. \text{ Statistical inference in hidden Markov models using k-segment constraints.} Journal of the American Statistical Association 111, no. 513 (2016): 200-215.

\item For predicting hospitalization associated with anticoagulant prescription patterns, we may need to look at a variety of reasons for hospitalization. These include medical diagnoses such as cancers, comorbidities, and other reasons. 

\item Consider a multi-category propensity score? $P(Y_{ij} | \bm X_{ij})$, where $Y_{ij} = $ AC fill at time $T_{ij}$, and $\bm X_{ij} = $covariate information up to time $T_{ij}$.

\end{itemize}


\end{document}
